% ------ Clase de documento ------
\documentclass[a4paper,10pt,oneside]{article}

%-------------------------------------Paquetes-----------------------------------------------------------------------
\usepackage[spanish]{babel}  % Traduce los textos a castellano
\usepackage[utf8]{inputenc}	% Permite escribir directamente áéíóúñ
\usepackage{t1enc}            	% Agrega caracteres extendidos al font
\usepackage{amsmath} 		%Permite imprimir mas opcciones matematicas
\usepackage{graphicx}		%Permite agregar imagenes al informe
\usepackage{multicol}  		%Permite dividir el texto en varias columnas
\usepackage{anysize}		%Permite modificar los margenes del documento
\usepackage{float} 			%Permite utilizar H para colocar las imagenes en un lugar especifico 
\usepackage{multirow}		%Permite dividir las tablas en subtablas
\usepackage{booktabs}		%Permiten manejar mejor el tamaño de las tablas
\usepackage{tabulary}		%Permiten manejar mejor el tamaño de las tablas
\usepackage{fancyhdr}		%Permite agregar encabezado y pie fancy

\usepackage{courier}		%
\usepackage{color}			%
\usepackage{listings}  		%Permite agregar codigo directamente sobre el documento
\usepackage{textcomp}

%---------------------------------------Definiciones propias---------------------------------------------------------
\newcommand{\grad}{\hspace{-2mm}$\phantom{a}^{\circ}$} %El º que no existe como comando
\newcommand{\oiint}{\displaystyle\bigcirc\!\!\!\!\!\!\!\!\int\!\!\!\!\!\int} %Integral doble cerrada
%------------------------------------------------------------------------------------------------------------------------

%-------------------------------------Configuracion De Codigo----------------------------------
\definecolor{dkgreen}{rgb}{0,0.6,0}
\definecolor{gray}{rgb}{0.5,0.5,0.5}
\definecolor{gray99}{gray}{.99}
\definecolor{gray95}{gray}{.95}
\definecolor{gray75}{gray}{.75}
\definecolor{gray50}{gray}{.50}
\definecolor{keywords_blue}{rgb}{0.13,0.13,1}
\definecolor{comments_green}{rgb}{0,0.5,0}
\definecolor{strings_red}{rgb}{0.9,0,0}
%\captionsetup[lstlisting]{format=listing,labelfont=style_labelfont,textfont=style_textfont}%

\lstset{
	aboveskip = {1.5\baselineskip},
	backgroundcolor = \color{gray99},
	basicstyle = \ttfamily\footnotesize,
	breakatwhitespace = true,   
	breaklines = true,
	captionpos = t,
	columns = fixed,
	commentstyle = \color{comments_green},
	escapeinside = {\%*}{*)}, 
	extendedchars = true,
	frame = lines,
	keywordstyle = \color{keywords_blue}\bfseries,
	language = C,                       
	numbers = left,
	numbersep = 5pt,
	numberstyle = \tiny\ttfamily\color{gray50},
	prebreak = \raisebox{0ex}[0ex][0ex]{\ensuremath{\hookleftarrow}},
	rulecolor = \color{gray75},
	showspaces = false,
	showstringspaces = false, 
	showtabs = false,
	stepnumber = 1,
	stringstyle = \color{strings_red},                                    
	tabsize = 2,
	title = \null, % Default value: title=\lstname
	upquote = true,                  
}


\newcommand{\captionlisting}[2][]{%
    \lstinputlisting[caption={\large{\detokenize{#2}}},#1]{#2}%
}

\renewcommand\lstlistingname{Archivo}
%%%%%%%%%%%%%%%%%%%%%%%%%%%%%%%%%%%%%%%%%%%%%%%%%%%

% ------ Configuración ------
\title{\textbf{66.20 Organización de Computadoras\\ Trabajo Práctico 0: \\ Infraestructura básica}}

\author{	Burdet Rodrigo, \textit{Padrón Nro. 93440}\\
            \texttt{ rodrigoburdet@gmail.com}\\\\
            Colangelo Federico, \textit{Padrón Nro. 89869}                     \\
            \texttt{ federico.colangelo@semperti.com}\\\\
            Manzano Matias, \textit{Padrón Nro. 83425}                     \\
            \texttt{ matsebman@gmail.com }\\\\[2.5ex]
            \normalsize{2do. Cuatrimestre de 2014}                       \\
			\normalsize{66.20 Organización de Computadoras}\\
            \normalsize{Facultad de Ingeniería, Universidad de Buenos Aires}            \\
       }
\date{\today}



% ----- Cuerpo del documento -----
\begin{document}
\maketitle

\thispagestyle{empty}

\newpage

\section{Objetivos}
    Familiarizarse con las herramientas de software que usaremos en los siguientes trabajos, implementando un programa (y su correspondiente documentación) que resuelva el problema piloto que presentaremos mas abajo.

\section{Resumen}
	En el presente trabajo, se implementó un algoritmo que permite graficar los conjuntos de Mandelbrot dados ciertos parámetros para permitir centrarnos en una región en particular de dicho conjunto.
	El programa fue realizado en c 99 en un entorno de desarrollo linux. Compilado en Linux y en un ambiente emulado NetBSD.

\section{Desarrollo}
	
\subsection{Paso 1: Configuración de Entorno de Desarrollo}
El primer paso fue configurar el entorno de desarrollo, de acuerdo a la guía facilitada por la cátedra. \\
Trabajamos con distribuciones Linux y con el GxEmul proporcionado por la cátedra, emulando un sistema NetBSD.	
\subsection{Paso 2: Implementación del programa}
El programa debe ejecutarse por línea de comando y la salida del mismo dependerá del valor de los argumentos con los que se lo haya invocado.
\subsubsection{Ingreso de parámetros}
El formato para invocar al programa es el siguiente:
\begin{center}
	\texttt{./tp0 [OPTIONS]}
\end{center}
Los parámetros válidos que puede recibir el programa son los siguientes: \\ 

\begin{ttfamily}
\begin{tabular}{lll}

\bf{-o,} & \bf{--output} &(Parámetro obligatorio. Especifica archivo de salida, - para stdout). \\
\bf{-r,} & \bf{--resolution} &(Resolución de la imagen de salida).\\
\bf{-c,} & \bf{--center} &(Centro de la imagen).\\
\bf{-w,} & \bf{--width} &(Ancho del rectángulo a dibujar).\\
\bf{-H,} & \bf{--height} &(Alto del rectángulo a dibujar).\\
\bf{-v,} & \bf{--version} 	&(Muestra la versión).\\
\bf{-h,} & \bf{--help} &(Muestra la ayuda).\\
\end{tabular}
\end{ttfamily}

\subsubsection{Interpretación de parámetros}
Para parsear los parámetros se usó la librería de GNU getopt, en particular se usó \texttt{getopt\_long} para permitir el pasaje de parámetros largos.
			
	\section{Compilación del programa}
	
	Para poder compilar el proyecto, se debe abrir una terminal Linux dentro del directorio donde se encuentra el código 
	fuente escrito en C, y ejecutar el siguiente comando:
	\begin{center}
	\texttt{gcc -Wall -std=c99  main.c -o tp0}\footnote{Requiere tener instalado el compilador \it{GCC}} 
	\end{center}
		
	 Esto generara un archivo ejecutable, llamado \textit{tp0}.

	\section{Compilación del programa en NetBSD}
	
	Para poder compilar el proyecto en NetBSD, se debe ejecutar el comando:
	
	\begin{center}
	\texttt{gcc -Wall -std=c99 -S -O0 main.c} 
	\end{center}
	
	
\section{Corridas de prueba y Mediciones}

	En las figuras que siguen a continuación se muestran los comandos utilizados para ejecutar el programa y se puede apreciar los resultados de las diferentes pruebas que realizamos.	
	\begin{figure}[H]
		\begin{center}
			\includegraphics[width=0.50\textwidth]{uno.png}
		\end{center}
		\caption{Llamada por defecto ./tp0 -o uno.pgm} \label{Figura 1}
	\end{figure}

	\begin{figure}[H]
		\begin{center}
			\includegraphics[width=0.80\textwidth]{dos.png}
		\end{center}
		\caption{Llamada haciendo zoom sobre la región centrada en (0.282 , -0.01) con cuadrado de 0.005 de lado. ./tp0 -c +0.282-0.01i -w 0.005 -H 0.005 -o dos.pgm } \label{Figura 2}
	\end{figure}

	\begin{figure}[H]
		\begin{center}
			\includegraphics[width=0.50\textwidth]{tres.png}
		\end{center}
		\caption{Llamada haciendo zoom sobre la región centrada en (0.296 , -0.02) con cuadrado de 0.003 de lado. ./tp0 -c +0.296-0.02i -w 0.003 -H 0.003 -o tres.pgm } \label{Figura 3}
	\end{figure}
	
	\newpage

\section{Conclusiones}	
	Como se enuncia en el objetivo de este trabajo práctico, aprendimos a instalar y manejar el GxEmul, a realizar transferencias de archivos en Linux, así como también compilar y ejecutar programas en el NetBSD. Por otro lado,  aprendimos a manejar y escribir informes en \LaTeX{}.
	De este modo, estamos preparados para que en los próximos trabajos prácticos, nos aboquemos directamente al desarrollo de los mismos.


\newpage 
\section{Código}
\subsection{main.c}
\lstset{ language = c }
\lstinputlisting[label=maincode]{../main.c}

\newpage 

\subsection{main.s}
\lstset{ language = [x86masm]assembler }
\lstinputlisting[label=mainmipscode]{../main.s}

\end{document}


